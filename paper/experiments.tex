The implementation of our ideas is ongoing and we do not have any results to share at this stage. We will embed our algorithm within DC-SVM where it will be used in the concatenation step, but we will also test it independently starting from random initializations for $\al$.

We expect that many nodes in a graph, e.g. those from distant clusters will satisfy Theorem and thus implementing this algorithm in Galois will result in considerable speedup over the serial version, depending on the dataset and kernel parameter $\sigma$. However, an important baseline is to consider randomly chosen points for simultaneous execution, i.e. random asynchronous execution. We expect that this procedure will decrease the objective as efficiently as our approach in the initial iterations, but as we have noted in a past assignment, this objective value will plateau at a high value for the rand-async baseline, and we expect that our algorithm will continue to decrease the value as efficiently past this stage eventually achieving an optimal value close to the serial baseline.

The second component of our project is the priority scheduling scheme. We expect that prioritizing updates will result in considerable speedup over random scheduling, since the change in the objective is higher in each iteration. We will also compare performance with different values of $\delta$. If possible, we would like to prove a theoretical result to demonstrate that this will be the case, but this is a low priority for this project.

We will also compare our algorithm against the multicore version of LibSVM and against other state-of-the-art solvers which are easily available.
