We ran two types of tests to validate that preexisiting models worked and were
comparable to our proposed sensing matrices that combined the two types of
models. In one, we simulate compressed sensing by creating sensing operators,
measuring an image and then solving a TV-minimization problem to attempt to
recover the image. We then compare the recovered image to the original to
determine the relative error. Concretely, we compute the relative error
$\varepsilon$ as

\begin{equation}
	\frac{||x^{\#} - x||_2}{||x||_2} = \varepsilon
\end{equation}

and compare this error for each of the possible methods considered. In the other
test, we were only interested in the time to compute the solution $x^{\#}$ for
the various types of methods. In particular, we wanted to show that the
structured random matrices are much more efficient than using a completely
random matrix with Gaussian i.i.d. entries. 


\begin{table}[h]
\begin{tabular}{lll|lllll}
	\textbf{Matrix Type}       & \textbf{Input Randomizer} & \textbf{Output Randomizer} & \textbf{Error} & \textbf{Time} & \\ \hline
	\multirow{2}{*}{Circulant} & \multirow{2}{*}{N/A}      & Sub                        & 0.0156         & 4.97s         & \\
	                           &                           & RPMS                       & 0.0159         & 7.80s         & \\ \hline


	\multirow{2}{*}{Toeplitz}  & \multirow{2}{*}{N/A}      & Sub                        & 0.0193         & 25.9s         & \\
	                           &                           & RPMS                       & 0.0170         & 16.9s         & \\ \hline

	\multirow{4}{*}{Fourier}   & \multirow{2}{*}{Local}    & Sub                        & 0.0183         & 7.89s         & \\
	                           &                           & RPMS                       & 0.0517         & 14.0s         & \\
	                           & \multirow{2}{*}{Global}   & Sub                        & 0.0166         & 4.98s         & \\
	                           &                           & RPMS                       & 0.0529         & 14.1s         & \\ \hline

	\multirow{4}{*}{Hadamard}  & \multirow{2}{*}{Local}    & Sub                        & 0.0167         & 552s          & \\
	                           &                           & RPMS                       & 0.0153         & 752s          & \\
	                           & \multirow{2}{*}{Global}   & Sub                        & 0.0157         & 544s          & \\
	                           &                           & RPMS                       & 0.0137         & 782s          & \\
\end{tabular}
\caption{Table of reconstruction error $\varepsilon$ and timing results for all it the structured sensing matrices described previously.}
\label{tab:errors}
\end{table}

We see in Table \ref{tab:errors} that the reconstruction errors for almost all
of the methods are comparable. However, in the case of a Fourier transform
sensing matrix with RPMS subsampling operator does not work well. In addation
to not achieving a solution as good as the othertested methods, it showed poor
convergence compared to the simpler subsampling methods in the sense that it
took longer to converge. We are not convinced that the RPMS method is working in
that case, but that it is functioning correctly for Circulant, Toeplitz and
Hadamard sensing matrices (where we are unaware of it having ever been used
before). 

\begin{table}[h]
\begin{tabular}{l|lllllll}
	\textbf{Matrix Type} & Gaussian & Circulant & Toeplitz & Fourier & Hadamard \\ \hline
	\textbf{Time}        & 12.2s    & 1.65s     & 4.24s    & 1.50s   & 102s*    \\
\end{tabular}
\label{tab:times}
\caption{ In this table we see how the time to compute the solution for a smaller problem using the random structured sensing operators compares with the time to compute the solution using a completely random matrix with Gaussian i.i.d. entries. *The fast Walsh-Hadamard transform in \textsc{Matlab} is written in \textsc{Matlab} and is slow.}
\end{table}


We see in Table \ref{tab:times} that the the structured random matrices are
able to compute the solution much more quickly than the completely random
matrix, with the exception of the operator using the fast Walsh-Hadamard
transform, which suffers from a slow implementation in \textsc{Matlab}. However,
the transform has the same complexity as the FFT and so we expect that if the
two were implemented similarly, that thet should exhibit similar performance.
The Toeplitz based sensing matrix is a little slower than the others using a
fast transform, but this is expected since performing a fast Toeplitz
matrix-vector multiplication requires using an FFT on a vector that is twice the
size of the one you wish to multiply. 

Overall the tests indicate the importance of the fast transforms for large
problems and the possibility of using the RPMS restriction operator with sensing
matrices other than circulant as originally suggested by
