Compressive sensing (CS) is a topic that attracted a lot of interest in the recent years.
The fundamental problem to solve is the basis pursuit problem, wherein we are trying to 
recover a length $n$ signal $z$ from $m$ measurements $y$ and $m \ll n$. In general 
this problem is underdetermined, but it is well posed when the solution $z$ is sparse 
in some basis $\Psi$. Then we let $\Psi x = z$, and assume $x$ is $S$-sparse, or has $S$ nonzero entries. We call $x$, and, by extension $z$, compressible if it can at least be can be well approximated using only $S \ll N$ coefficients. Then we define the Basis pursuit problem as follows:

\begin{equation}
\min ||x||_{l_1} \quad \text{ subject to } \quad \Phi \Psi x=y
\end{equation}


where $\Psi$ is the $n \times n$ orthogonal matrix corresponding to sparsifying basis
above and $\Phi$ is an $m \times n$ sensing matrix and as above we take $m\ll n$. 

\subsection*{Structured matrix}
\textbf{Toeplitz} and \textbf{Circulant} matrices have the forms, respectively,
\\

$$
T = \begin{bmatrix}
t_{1}    & t_{2}    & ...    & t_{n}   \\[0.3em]
t_{n+1}  & t_{1}    & ...    & t_{n-1} \\[0.3em]
\ddots   & \ddots   & \ddots &         \\[0.3em]
t_{2n-1} & t_{2n-2} & ...    & t_{1}         
\end{bmatrix}
\qquad \text{and} \qquad
C = \begin{bmatrix}
t_{1}  & t_{2}  & ...    & t_{n}   \\[0.3em]
t_{n}  & t_{1}  & ...    & t_{n-1} \\[0.3em]
\ddots & \ddots & \ddots &         \\[0.3em]
t_{2}  & t_{3}  & ...    & t_{1}        
\end{bmatrix} 
$$
\\
where every left-to-right descending diagonal is constant, i.e. $T_{i,j}=T_{i+1,j+1}.$ If $T$ satisfies the additional property that $t_{i}=t_{n+i}, \forall i,$ it is a circulant matrix $C$.
	Any Circulant matrix can be diagonalized by a Fourier transform \cite{wotao}, i.e. obeying
	$$ C=\frac{1}{\sqrt{n}} F^* \Sigma F $$ with $F$ as the discrete Fourier matrix
	$F_{t,w}=e^{-i\; 2\pi(t-1)(w-1)/n}, \qquad 1 \le t,w \le n$.
	
	Note this factorization ensures that $C$ will be orthogonal, i.e.
	$$ C^*C=\dfrac{{1}}{n} F^* \Sigma F F^* \Sigma F = nI, $$
	since $FF^*=F^*F=nI$ and $\Sigma^* \Sigma = I$ (if we choose the diagonal of $\Sigma$ 
	to have magnitude $1$).
	
According to \cite{romberg2009}, the diagonal $\sigma_{w}$ of $\Sigma$ was generated enforcing conjugate symmetry to ensure that the resulting circulant matrix is real-valued. The following rules were used in construction:
    	
	$w=1\qquad \qquad \qquad : \; \sigma_{1} \sim \pm$ 1 with equal probability,
	\\[1em]
	$2 \le w < n/2+1 \; \; : \; \sigma_{w}=e^{i\theta w}, where \; {\theta}_{w} \sim Uniform(0,2\pi) $
	\\[1em]
	$w=n/2+1 \qquad \; \; \, : \; \sigma_{n/2+1} \sim \pm 1 $ with equal probability 
	\\[1em]
	$n/2+2 \le w \le n \; \; : \; \sigma_{w}=\sigma^{*}_{n-w+2}$, conjugate of  $\sigma_{n-w+2}$.
    
	Now we will demonstrate how Toeplitz matrices can be embedded in larger systems that 
	are circulant. Breaking up the $2n-1$ degrees of freedom into two vectors as shown 
	below,
	$$
	T_{n} = \left[ \: 
	\begin{array}{*{4}{c}}
	\cline{1-4}
	\multicolumn{1}{|c}{t_{1}} & t_{2} & ...& \multicolumn{1}{c|}{t_{n}}           \\[0.3em]
	\cline{1-4}
	t_{n+1} & t_{1} & ... & t_{n-1} \\[0.3em]
	\ddots &\ddots & \ddots &    \\[0.3em]
	t_{2n-1} & t_{2n-2} & ... & t_{1}      \\[0.3em]    
	\end{array}
	\right]
	=
	\left[ \:
	\begin{array}{*{4}{c}}
	t_{1} & t_{2} & ...& t_{n}        \\[0.3em]
	\cline{1-1}
	\multicolumn{1}{|c|}{t_{n+1}} & t_{1} & ... & t_{n-1} \\[0.3em]
	\multicolumn{1}{|c|}{\ddots} & \ddots & \ddots &    \\[0.3em]
	\multicolumn{1}{|c|}{t_{2n-1}} & t_{2n-2} & ... & t_{1}          \\[0.3em]
	\cline{1-1}
	\end{array}
	\right]
	$$
	We now take the horizontal vector and leave it in place, and add a placeholder $0$. 
	Finally we add the vertical vector above after it has been flipped and transposed to 
	the first row of our new matrix. If we circulate these entries, the resulting circulant
	matrix has $T_n$ in the top corner. 
	$$
	C_{2n} =\begin{bmatrix}
	t_{1}   & t_{2} & ...    & t_{n}  & 0   & t_{2n-1} & ... &  t_{n+1}   \\[0.3em]
	t_{n+1} & t_{1} & ...    & t_{n-1}& ... & 0  & ... &          \\[0.3em]
	\ddots  &       & \ddots &        & \ddots & & &\ddots    \\[0.3em]
	\cdots  &       & \cdots &       & \cdots & & & \cdots
	\end{bmatrix}
	=
	\begin{bmatrix}
	T_{n} & B_{n}            \\[0.3em]
	B_{n} & T_{n}   
	\end{bmatrix}
	$$	
	Thus, we can perform a fast circulant multiply on this system and extract the value we
	want, $T_n x$, from the top half of the resulting vector:
	$$
	\begin{bmatrix} C_{2n} 
	\end{bmatrix} 
	\begin{bmatrix} x \\[0.3em] 0 \end{bmatrix} =
	\begin{bmatrix}
	T_{n} & B_{n}            \\[0.3em]
	B_{n} & T_{n}   
\end{bmatrix} \begin{bmatrix} x \\[0.3em] 0 \end{bmatrix}
= \begin{bmatrix} T_n x \\[0.3em] B_n x \end{bmatrix}
 $$

	\textbf{Hadamard} matrix is a square matrix whose entries are either $+1$ or $-1$ and whose rows are mutually orthogonal.For example, one of way to construct Hadamard matrix is shown here :
	
	$$
	H_{1} = \begin{bmatrix}
	1
	\end{bmatrix}
	\qquad
	H_{2} = \begin{bmatrix}
	1 & 1           \\[0.3em]
	1& -1
	\end{bmatrix}	
	\qquad
	H_{4} = \begin{bmatrix}
	H_{2} & H_{2}           \\[0.3em]
	H_{2}& -H_{2}
	\end{bmatrix}	
	$$
	$$
	\centering
	\qquad
	H_{2^{n}} = \begin{bmatrix}
	H_{2^{n-1}} & H_{2^{n-1}}           \\[0.3em]
	H_{2^{n-1}}& -H_{2^{n-1}}            
	\end{bmatrix}	
	$$
These matrices define the Hadamard transform, which is a generalized Fourier transform 
that we will use in our experiments section. 
