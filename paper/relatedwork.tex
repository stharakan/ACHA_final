\paragraph{Related Work}
Our work expands on the work presented in two previous papers, \cite{doetal} and \cite{romberg2009}, each of which proposed a different random sensing matrix framework. \cite{wotao} motivated a different random Toeplitz structure, which we will discuss in the next section. \cite{doetal} characterized random sensing matrices as the product of three matrices: $DFR$. The structure of each is outlined below:

\begin{itemize}
	\item $D\in \mathbb{R}^{M\times N}$ (output randomizer) is a subsampling matrix/operator which randomly selects a subset of the entries of a vector, possibly permuting it. We call this subsampling.
	\item $F\in\mathbb{C}^{N\times N}$ is an orthonormal matrix. We usually use a fast transform like the FFT or fast Walsh-Hadamard transform to ensure fast matrix-vector multiplies.
	\item $R\in\mathbb{R}^{N\times N}$ (input randomizer) is one of the following:
		\begin{itemize}
			\item A uniform random permutation matrix, corresponding to global randomization
			\item A diagonal matrix whose entries take on $\pm 1$ with probability $1/2$, or Bernoulli random variables. We refer to this as local randomization.
		\end{itemize}
\end{itemize}

While the approach in \cite{romberg2009} was similar, it notably had no input randomizer. This was due to fact that the matrix was a circulant matrix generated from a random phase, and then coupled with a random selection operator. The random selection operator here played the same role as $D$ above. However, \cite{romberg2009} proposed another method he called randomly premodulated summation (RPMS). Instead of selecting rows of a vector at random, RPMS adds chunks of the vector together to create a smaller vector. To ensure randomization, the entries of the original vector are added after being multiplied by a Bernoulli random variable. While the theoretical bounds for RPMS were worse by a log factor, it is important to note that it is much easier to implement in hardware.

\cite{wotao} presented studies on circulant and toeplitz matrices, but implemented them by taking Bernoulli variables along the first row and/or column. This randomization is different than the one in \cite{romberg2009} as it is more restrictive in the values it can take on. 

\paragraph{Our Contributions}
Our work can be summarized as follows
\begin{itemize}
	\item We have combined the structures \cite{doetal} and the subsampling operator 
		of \cite{romberg2009} to propose a new type of structured random matrix which offers 
		comprable practical performance. In the language of the previous works, this refers
		to making $D$ matrix different from the one used in \cite{doetal}, where the new $D$
		represents RPMS. We have run some tests with this structure and compared it to 
		our results on other types of structured matrices.
	\item We implemented a randomized Toeplitz matrix that built on the random structure
		used by \cite{romberg2009} for circulant matrices by expanding the system to size
		$2N$. A Toeplitz matrix product can be computed efficiently using this notion that 
		it is embedded in a circulant system of double the size. Using this knowledge, we 
		could combine the phase generation outlined in \cite{romberg2009} to generate a 
		Toeplitz matrix in a different way than how it is done in \cite{wotao}. We also ran 
		tests on these to quantify their effectiveness. 
\end{itemize}
