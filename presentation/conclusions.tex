%%%%%%%%%%%%%%%%%%%%%%%%%%%%%%%%%%%%%%
%% Frame
%%%%%%%%%%%%%%%%%%%%%%%%%%%%%%%%%%%%%%
\begin{frame}[t]
	\frametitle{Our contributions}
	\framesubtitle{~~}  %% needed for proper positioning of the logo ...

	Our work can be summarized as follows
	\begin{itemize}
		\item We have combined the structures Do et al. (2012) and the subsampling operator 
			of Romberg (2008) to propose a new type of structured random matrix which offers 
			comprable practical performance
		\item We implemented a randomized Toeplitz matrix that built on the random structure
			used by Romberg (2008) for circulant matrices by expanding the system to size $2N$.
	\end{itemize}

\end{frame}



%%%%%%%%%%%%%%%%%%%%%%%%%%%%%%%%%%%%%%
%% Frame
%%%%%%%%%%%%%%%%%%%%%%%%%%%%%%%%%%%%%%
\begin{frame}[t]
	\frametitle{Future work}
	\framesubtitle{~~}  %% needed for proper positioning of the logo ...

	Looking ahead, we have the following questions for areas of future research
	\begin{itemize}
		\item We would like to see empirically how sparsity affects each of the methods we 
			have looked at to compare with the theory (comparing to the $S\log{N}$ bound. In 
			Yin et al (2010), they had some results, but not for most of the matrices we 
			compared here. 
		\item We did most minimizations in the $TV$ norm, but we would like to see an 
			implementation of an ADMM (alternating direction method of multipliers) on an
			objective function containing both the $L_1$ and the $TV$ norms. Yin et al (2010)
			implemented this, but not with the randomizations discussed here. 
	\end{itemize}

\end{frame}

%%%%%%%%%%%%%%%%%%%%%%%%%%%%%%%%%%%%%%
%% Frame
%%%%%%%%%%%%%%%%%%%%%%%%%%%%%%%%%%%%%%
\begin{frame}[t]
	\frametitle{Thank you}
	\framesubtitle{~~}  %% needed for proper positioning of the logo ...
	Questions?
\end{frame}

