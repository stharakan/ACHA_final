%%%%%%%%%%%%%%%%%%%%%%%%%%%%%%%%%%%%%%
%% Frame
%%%%%%%%%%%%%%%%%%%%%%%%%%%%%%%%%%%%%%

\begin{frame}[t]
\frametitle{Introduction}
\framesubtitle{~~}  %% needed for proper positioning of the logo ...

The Basis pursuit problem 
\begin{equation}
min  \;||x||_{l_1} \qquad   subject \; to \; \Phi \Psi x=y
\end{equation}

\small
With $\Psi$ as an $n \times n$ orthogonal representation matrix, i.e. $t_{1}$ is $S$-sparse in $\Psi$ if we can decompose $t_0$ as $t_0=\Psi x_0 $ , where $x_0$ has at most $S$ non-zero components.
\\
\normalfont
\textbf{Toeplitz} and \textbf{Circulant} matrices have the forms, respectively,
\\


$$
T = \begin{bmatrix}
	t_{1} & t_{2} & ...& t_{n}           \\[0.3em]
	t_{n+1} & t_{1} & ... & t_{n-1} \\[0.3em]
	\ddots &\ddots & \ddots &    \\[0.3em]
	t_{2n-1} & t_{2n-2}& ... & t_{1}         
\end{bmatrix}
\qquad and \qquad
C = \begin{bmatrix}
t_{1} & t_{2} & ...& t_{n}           \\[0.3em]
t_{n} & t_{1} & ... & t_{n-1} \\[0.3em]
\ddots &\ddots & \ddots &      \\[0.3em]
t_{2} & t_{3}& ... & t_{1}        
\end{bmatrix} 
$$
\end{frame}
%%%%%%%%%%%%%%%%%%%%%%%%%%%%%%%%%%%%%
%% Frame
%%%%%%%%%%%%%%%%%%%%%%%%%%%%%%%%%%%%%%
\begin{frame}[t]
\frametitle{Structured random matrices}
\framesubtitle{~~}  %% needed for proper positioning of the logo ...
	Any Circulant matrix can be diagonalized by a Fourier transform, i.e. obeying
	$$ C=\frac{1}{\sqrt{n}} F^* \Sigma F $$ with $F$ as the discrete Fourier matrix
	$F_{t,w}=e^{-i\; 2\pi(t-1)(w-1)/n}, \qquad 1 \le t,w \le n$
$$
\Sigma = \begin{bmatrix}
	\sigma_{1} & 0 & ...& 0           \\[0.3em]
	0 & \sigma_{2} & ... & 0 \\[0.3em]
	\ddots &\ddots & \ddots &      \\[0.3em]
	0 & 0 & ... & \sigma_{n}        
\end{bmatrix} $$
a diagonal matrix whose entries are unit magnitude complex numbers with random phases.
\end{frame}


%%%%%%%%%%%%%%%%%%%%%%%%%%%%%%%%%%%%%
%% Frame
%%%%%%%%%%%%%%%%%%%%%%%%%%%%%%%%%%%%%%
\begin{frame}[t]
	\frametitle{Structured random matrices}
	\framesubtitle{~~}  %% needed for proper positioning of the logo ...
	We generate $\sigma_{w}$ as follows:
    \\[1em]
    
    $w=1\qquad \qquad \qquad : \; \sigma_{1} \sim \pm$ 1 with equal probability,
    \\[1em]
    $2 \le w < n/2+1 \; \; : \; \sigma_{w}=e^{i\theta w}, where \; {\theta}_{w} \sim Uniform(0,2\pi) $
    \\[1em]
    $w=n/2+1 \qquad \; \; \, : \; \sigma_{n/2+1} \sim \pm 1 $ with equal probability 
    \\[1em]
    
    $n/2+2 \le w \le n \; \; : \; \sigma_{w}=\sigma^{*}_{n-w+2}$, conjugate of  $\sigma_{n-w+2}$.
    

\end{frame}

%%%%%%%%%%%%%%%%%%%%%%%%%%%%%%%%%%%%%%
%% Frame
%%%%%%%%%%%%%%%%%%%%%%%%%%
%%%%%%%%%%%%
\begin{frame}[t]
	\frametitle{Structured random matrices}
	\framesubtitle{~~}  %% needed for proper positioning of the logo ...

\left
$$
T_{n} = \left[ \: 
% \: serves as a spacer between the left-hand bracket of 
% the matrix and the left-hand side of the inner frame
\begin{array}{*{13}{c}}
\cline{1-4}
\multicolumn{1}{|c}{t_{1}} & t_{2} & ...& \multicolumn{1}{c|}{t_{n}}           \\[0.3em]
\cline{1-4}
	t_{n+1} & t_{1} & ... & t_{n-1} \\[0.3em]
	\ddots &\ddots & \ddots &    \\[0.3em]
	t_{2n-1} & t_{2n-2}& ... & t_{1}      \\[0.3em]    
	\end{array}
\right]
$$
$$
T_{n} = \left[ \:
\begin{array}{*{13}{c}}

t_{1} & t_{2} & ...& t_{n}        \\[0.3em]
\cline{1-1}
\multicolumn{1}{|c|}{t_{n+1}} & t_{1} & ... & t_{n-1} \\[0.3em]
\multicolumn{1}{|c|}{\ddots} &\ddots & \ddots &    \\[0.3em]
\multicolumn{1}{|c|}{t_{2n-1}} & t_{2n-2}& ... & t_{1}      \\[0.3em]    
\cline{1-1}
\end{array}
\right]
$$
\\
$$
C_{2n} =\begin{bmatrix}
	t_{1} & t_{2} & ...& t_{n} & 0 & t_{2n-1} & ...&  t_{n+1}           \\[0.3em]
    t_{n+1} & t_{1}  & ... & t_{n-1} & ... & 0 & ...    \\[0.3em]
	\ddots &  & \ddots & & \ddots &  & \ddots \\[0.3em]


	
\end{bmatrix}
 =\begin{bmatrix}
	T_{n} & B_{n}            \\[0.3em]
	B_{n} & T_{n}   \\[0.3em]
\end{bmatrix}
$$
	
\end{frame}

%%%%%%%%%%%%%%%%%%%%%%%%%%%%%%%%%%%%%
%% Frame
%%%%%%%%%%%%%%%%%%%%%%%%%%%%%%%%%%%%%%
\begin{frame}[t]
	\frametitle{Structured random matrices}
	\framesubtitle{~~}  %% needed for proper positioning of the logo ...
	 \textbf{Hadamard} matrix is a square matrix whose entries are either $+1$ or $−1$ and whose rows are mutually orthogonal.
\begin{figure}[h]
\centering
\includegraphics[width=0.7\linewidth]{HadamardMatrices_800}
\caption{}
\label{fig:HadamardMatrices_800}
\end{figure}

\end{frame}
%%%%%%%%%%%%%%%%%%%%%%%%%%%%%%%%%%%%%%
%% Frame
%%%%%%%%%%%%%%%%%%%%%%%%%%%%%%%%%%%%%%
\begin{frame}[t]
	\frametitle{Structured random matrices}
	\framesubtitle{~~}  %% needed for proper positioning of the logo ...
	An equivalent definition of the Hadamard matrices is given by 
	$$ H_{n} H_{n}^T = n  I_{n} $$
	where $I_{n}$ is the $n \times n$ identity matrix.
$$
	H_{1} = \begin{bmatrix}
		1
		\end{bmatrix}
\qquad
    H_{2} = \begin{bmatrix}
 		1 & 1           \\[0.3em]
		1& -1
			\end{bmatrix}	
\qquad
	H_{4} = \begin{bmatrix}
		H_{2} & H_{2}           \\[0.3em]
		H_{2}& -H_{2}
	\end{bmatrix}	
$$
$$
\centering
\qquad
H_{2^{n}} = \begin{bmatrix}
H_{2^{n-1}} & H_{2^{n-1}}           \\[0.3em]
H_{2^{n-1}}& -H_{2^{n-1}}            
\end{bmatrix}	
$$
\end{frame}
       
