\documentclass[mathserif,18pt,xcolor=table]{beamer}

\usepackage {bbm}
\usepackage {textpos}
\usepackage{tikz}
\usepackage{hyperref}
\usepackage{color}
\usepackage{amsmath}
\usepackage{amssymb}
\usepackage{soul}
\usepackage[compatibility=false]{caption}
\usepackage{subcaption}
\usepackage{multimedia}



\definecolor{utorange}{RGB}{203,96,21}
\definecolor{utblack}{RGB}{99,102,106}
\definecolor{utbrown}{RGB}{110,98,89}
\definecolor{utsecbrown}{RGB}{217,200,158}
\definecolor{utsecgreen}{RGB}{208,222,187}
\definecolor{utsecblue}{RGB}{127,169,174}

\newcommand{\norm}[1]{\left\lVert #1 \right\rVert}

\mode<presentation>
{
  % \usetheme{Pittsburgh}   
  \usetheme{Boadilla}  
  \usefonttheme[onlymath]{serif}

  \setbeamercovered{invisible}
  \setbeamertemplate{navigation symbols}{}

  % Color Theme 
    \setbeamercolor{normal text}{bg=white,fg=utblack}
  \setbeamercolor{structure}{fg=utorange}

  \setbeamercolor{alerted text}{fg=red!85!black}

  \setbeamercolor{item projected}{use=item,fg=black,bg=item.fg!35}

  \setbeamercolor*{palette primary}{use=structure,fg=white, bg=utorange}
  \setbeamercolor*{palette secondary}{use=structure,bg=utsecbrown}
  \setbeamercolor*{palette tertiary}{use=structure,bg=utsecgreen}
  \setbeamercolor*{palette quaternary}{use=structure,fg=structure.fg,bg=utsecblue}

  % \setbeamercolor*{frametitle}{use=structure,fg=utorange, bg=utsecbrown}
  \setbeamercolor*{framesubtitle}{fg=utbrown}

  \setbeamercolor*{block title}{parent=structure,fg=black,bg=utsecgreen}
  \setbeamercolor*{block body}{fg=black,bg=utblack!10}
  \setbeamercolor*{block title alerted}{parent=alerted text,bg=black!15}
  \setbeamercolor*{block title example}{parent=example text,bg=black!15}

  \setbeamerfont{framesubtitle}{size=\small}
}

\usepackage[orientation=landscape,size=custom,width=16,height=9.75,scale=0.5,debug]{beamerposter}
% \usepackage[orientation=landscape,size=custom,width=16,height=9,scale=0.5,debug]{beamerposter}


\makeatletter
\setbeamertemplate{footline}
{
  \leavevmode%
    \hbox{%
      \begin{beamercolorbox}[wd=.333333\paperwidth,ht=2.25ex,dp=1ex,center]{author in head/foot}%
        \usebeamerfont{author in head/foot}\insertshortauthor%~~\beamer@ifempty{\insertshortinstitute}{}{(\insertshortinstitute)}
      \end{beamercolorbox}%
        \begin{beamercolorbox}[wd=.5\paperwidth,ht=2.25ex,dp=1ex,center]{title in head/foot}%
        \usebeamerfont{title in head/foot}\insertshorttitle
        \end{beamercolorbox}%
        \begin{beamercolorbox}[wd=.166666\paperwidth,ht=2.25ex,dp=1ex,right]{date in head/foot}%
        \usebeamerfont{date in head/foot}\insertshortdate{}\hspace*{2em}
        \insertframenumber{} / \inserttotalframenumber\hspace*{2ex} 
      \end{beamercolorbox}}%
        \vskip0pt%
}
\makeatother

\usepackage{kerkis}
\usepackage[T1]{fontenc}
\usepackage[protrusion=true,expansion=true]{microtype}
\usepackage{amsmath}


\renewcommand*{\thefootnote}{\fnsymbol{footnote}}


\pgfdeclareimage[height=1.0cm]{utbig}{logos/UTWordmark.pdf}
\pgfdeclareimage[height=0.6cm]{ut}{logos/UTWordmark.pdf}

\title{Stable Signal Recovery from Incomplete and Inaccurate Measurements}
\subtitle{Paper by Candes, Romberg and Tao}
\author[Keith Kelly, Sameer~Tharakan, Yerlan~Amanbek]{ {Keith~Kelly, Sameer~Tharakan, Yerlan~Amanbek} \\  
}

\institute{Institute for Computational Engineering \& Sciences\\ \mbox{}  \\  \pgfuseimage{utbig} }
\date[ACHA]

\begin{document}

\tikzstyle{block} = [rectangle, draw, rounded corners, shade, top color=white, text width=5em,
  bottom color=blue!50!black!20, draw=blue!40!black!60, very thick, text centered, minimum height=4em]
  \tikzstyle{line} = [draw, -latex']
  \tikzstyle{cloud} = [draw, ellipse,top color=white, bottom color=red!20, node distance=2cm, minimum height=2em]


  \beamertemplateballitem
  %\beamertemplatetransparentcoveredhigh

  \frame{\titlepage}

  \addtobeamertemplate{frametitle}{}{%
      \begin{textblock*}{100mm}(0.95\textwidth,-0.75cm)
    \pgfuseimage{ut}
    \end{textblock*}
  }


\section{Introduction}

% ------------------------------------------------------------
\begin{frame}[t]
\frametitle{Intro}
\framesubtitle{~~}  %% needed for proper positioning of the logo ...

Slides online at

\href{http://users.ices.utexas.edu/~keith/}{\textcolor{blue}{http://users.ices.utexas.edu/\textasciitilde keith/}}

$$
$$

Get the code online at

\href{https://github.com/kwkelly/ACHA}{\textcolor{blue}{https://github.com/kwkelly/ACHA}}


\end{frame}


%%%%%%%%%%%%%%%%%%%%%%%%%%%%%%%%%%%%%%
%% Frame
%%%%%%%%%%%%%%%%%%%%%%%%%%%%%%%%%%%%%%

\begin{frame}[t]
\frametitle{1D examples}
\framesubtitle{~~}  %% needed for proper positioning of the logo ...

Given an exactly sparse or a compressible signal, we solve

\begin{align*}
	\min \norm{x}_{\ell_1} \text{ subject to } \norm{Ax - y}_{\ell_2} \leq \epsilon
\end{align*}

with added Gaussian white noise with standard deviation $\sigma\in [0.01,0.5]$.


\end{frame}

%%%%%%%%%%%%%%%%%%%%%%%%%%%%%%%%%%%%%%
%% Frame
%%%%%%%%%%%%%%%%%%%%%%%%%%%%%%%%%%%%%%

\begin{frame}[t]
\frametitle{1D examples - Sparse Signal}
\framesubtitle{~~}  %% needed for proper positioning of the logo ...


\centering
Sparse signal left, recovered signal right, $\sigma = 0.5$.
\includegraphics[scale=0.4]{fig/paperSparseSignalrec.png}
\\
\includegraphics[scale=0.3]{fig/paperSparseSignalTable.png}


\end{frame}





%%%%%%%%%%%%%%%%%%%%%%%%%%%%%%%%%%%%%%
%% Frame
%%%%%%%%%%%%%%%%%%%%%%%%%%%%%%%%%%%%%%

\begin{frame}[t]
\frametitle{1D examples - Compressible Signal}
\framesubtitle{~~}  %% needed for proper positioning of the logo ...


\centering
Compressible signal left, recovered signal right, $\sigma = 0.5$.
\includegraphics[scale=0.4]{fig/paperCompSignalrec.png}
\\
\includegraphics[scale=0.3]{fig/paperCompSignalTable.png}

\end{frame}


%%%%%%%%%%%%%%%%%%%%%%%%%%%%%%%%%%%%%%
%% Frame
%%%%%%%%%%%%%%%%%%%%%%%%%%%%%%%%%%%%%%
\begin{frame}[t]
\frametitle{2D examples}
\framesubtitle{~~}  %% needed for proper positioning of the logo ...


In the 2D case we are solving a different problem, this time we solve
\begin{align*}
	\min \norm{x}_{TV} \text{ subject to } \norm{Ax = y}_{\ell_2} \leq \epsilon
\end{align*}

The authors show in another paper that images with sparse gradients can be exactly recovered by solving the the above with $\epsilon = 0$.

\end{frame}



%%%%%%%%%%%%%%%%%%%%%%%%%%%%%%%%%%%%%%
%% Frame
%%%%%%%%%%%%%%%%%%%%%%%%%%%%%%%%%%%%%%

\begin{frame}[t]
\frametitle{2D examples - Boats}
\framesubtitle{~~}  %% needed for proper positioning of the logo ...


\quad\qquad Original Image \qquad  Gaussian Noise Recovery \hspace{2pt} Round-off Error Recovery
\centering
\includegraphics[scale=0.3]{fig/boats.png}
\\
\includegraphics[scale=0.3]{fig/boatsTable.png}

\end{frame}

%%%%%%%%%%%% END PAPER EXAMPLES %%%%%%%%%%%%%%%%%%%%%%%%%%%%%%%%%%%%

\section{Examples}

%%%%%%%%%%%%%%%%%%%%%%%%%%%%%%%%%%%%%%
%% Frame
%%%%%%%%%%%%%%%%%%%%%%%%%%%%%%%%%%%%%%

\begin{frame}[t]
\frametitle{More Examples!}
\framesubtitle{~~}  %% needed for proper positioning of the logo ...


On to some of our own examples\dots


\end{frame}




%%%%%%%%%%%%%%%%%%%%%%%%%%%%%%%%%%%%%%
%% Frame
%%%%%%%%%%%%%%%%%%%%%%%%%%%%%%%%%%%%%%
\begin{frame}[t]
\frametitle{2D examples}
\framesubtitle{~~}  %% needed for proper positioning of the logo ...

In the 2D case we are solving the TV problem
\begin{align*}
	\min \norm{x}_{TV} \text{ subject to } \norm{Ax = y}_{\ell_2} \leq \epsilon.
\end{align*}
We use the package \href{http://users.ece.gatech.edu/~justin/l1magic/}{\textcolor{blue}{$\ell_1$-MAGIC}} to perform the optimization.

\end{frame}



%%%%%%%%%%%%%%%%%%%%%%%%%%%%%%%%%%%%%%
%% Frame
%%%%%%%%%%%%%%%%%%%%%%%%%%%%%%%%%%%%%%
\begin{frame}[t]
\frametitle{2D examples - Lena}
\framesubtitle{~~}  %% needed for proper positioning of the logo ...

We have the lovely Lena

\centering
\includegraphics[scale=0.3]{fig/lena512.pdf}

\end{frame}

%%%%%%%%%%%%%%%%%%%%%%%%%%%%%%%%%%%%%%
%% Frame
%%%%%%%%%%%%%%%%%%%%%%%%%%%%%%%%%%%%%%
\begin{frame}[t]
	\frametitle{2D examples - \st{Lena} Sameer}
\framesubtitle{~~}  %% needed for proper positioning of the logo ...

We have the lovely \st{Lena} Sameer

\centering
\includegraphics[scale=0.3]{fig/sameer512.pdf}

\end{frame}


%%%%%%%%%%%%%%%%%%%%%%%%%%%%%%%%%%%%%%
%% Frame
%%%%%%%%%%%%%%%%%%%%%%%%%%%%%%%%%%%%%%
\begin{frame}[t]
	\frametitle{2D examples - \st{Lena} Sameer}
\framesubtitle{~~}  %% needed for proper positioning of the logo ...
 Does Sameer's face have a sparse gradient?
\pause
Yes it does.

\centering
\includegraphics[scale=0.3]{fig/sameerGradient.pdf}

\end{frame}


%%%%%%%%%%%%%%%%%%%%%%%%%%%%%%%%%%%%%%
%% Frame
%%%%%%%%%%%%%%%%%%%%%%%%%%%%%%%%%%%%%%
\begin{frame}[t]
	\frametitle{2D examples - \st{Lena} Sameer}
\framesubtitle{~~}  %% needed for proper positioning of the logo ...

Indeed, we see that his gradient appears to follow a power law, which suggests that his face is compressible

\centering
\includegraphics[scale=0.3]{fig/sameerGradPlot.pdf}

\end{frame}


%%%%%%%%%%%%%%%%%%%%%%%%%%%%%%%%%%%%%%
%% Frame
%%%%%%%%%%%%%%%%%%%%%%%%%%%%%%%%%%%%%%
\begin{frame}[t]
	\frametitle{2D examples - \st{Lena} Sameer}

Using $512^2/3$ samples, and a scrambled fourier transform ensemble as our sampling matrix, we can reconstruct Sameer's
face using the fact that his gradient is compressible. In this case we get a reconstruction error $\norm{x^\# -x_0}_{\ell_2} = 0.0155$.
\begin{figure}
\centering
\includegraphics[scale=0.6]{fig/sameerTVrec.pdf}
\end{figure}


\end{frame}



%%%%%%%%%%%%%%%%%%%%%%%%%%%%%%%%%%%%%%
%% Frame
%%%%%%%%%%%%%%%%%%%%%%%%%%%%%%%%%%%%%%
\begin{frame}[t]
	\frametitle{2D examples - \st{Lena} Sameer}

	But what if we have noisy measurements? Using an SNR of 4.5 ($\sigma \approx 1.4\times10^{-4}$) we can still reconstruct
	Sameer's face, but it is not as good. In this case we get a reconstruction error $\norm{x^\# -x_0}_{\ell_2} = 0.0908$.
\pause

\begin{figure}
\centering
\includegraphics[scale=0.7]{fig/sameerTVsnr45.pdf}
\end{figure}



\end{frame}


%%%%%%%%%%%%%%%%%%%%%%%%%%%%%%%%%%%%%%
%% Frame
%%%%%%%%%%%%%%%%%%%%%%%%%%%%%%%%%%%%%%
\begin{frame}[t]
	\frametitle{2D examples - \st{Lena} Sameer}

How much data do we really need? Using an SNR of 4.5 we did some tests.
\pause

\begin{table}[h]
\begin{tabular}{l|l|l|l|l|l|l}
Observations                 & $mn/3$ & $mn/4$ & $mn/5$ & $mn/10$ & $mn/20$ & $mn/100$ \\ \hline
$\norm{x^\# - x_0}_{\ell_2}$ & 0.0908 & 0.0894 & 0.0905 & 0.0989  & 0.1169  & 0.2050   \\
\end{tabular}
\end{table}
\pause

\begin{figure}
\centering
\includegraphics[scale=0.5]{fig/sameerTV100.pdf}
\caption{Reconstruction with $mn/100$ samples}
\end{figure}

\end{frame}


%%%%%%%%%%%%%%%%%%%%%%%%%%%%%%%%%%%%%%
%% Frame
%%%%%%%%%%%%%%%%%%%%%%%%%%%%%%%%%%%%%%
\begin{frame}[t]
	\frametitle{2D examples - \st{Lena} Sameer}

And what happens if we use the $\ell_1$ norm in the frequency domain instead of the TV norm? Let's try solving this problem instead:

\begin{align*}
	\min \norm{\alpha}_{\ell_1} \text{ subject to } \norm{A\alpha - y}_{\ell_2} \leq \epsilon.
\end{align*}

where $\alpha = \text{dct}(x)$

With an SNR of 5000 and an image size to sample size ratio of 3.



\end{frame}

%%%%%%%%%%%%%%%%%%%%%%%%%%%%%%%%%%%%%%
%% Frame
%%%%%%%%%%%%%%%%%%%%%%%%%%%%%%%%%%%%%%
\begin{frame}[t]
	\frametitle{2D examples - \st{Lena} Sameer}

Well the DCT looks kind of sparse (note that we are looking at the log so we can see anything at all, the unscaled version is basically all black).

\begin{figure}
\centering
\includegraphics[scale=0.2]{fig/sameerdct.pdf}
\end{figure}

\end{frame}


%%%%%%%%%%%%%%%%%%%%%%%%%%%%%%%%%%%%%%
%% Frame
%%%%%%%%%%%%%%%%%%%%%%%%%%%%%%%%%%%%%%
\begin{frame}[t]
	\frametitle{2D examples - \st{Lena} Sameer}

Well the DCT looks kind of sparse.

\begin{figure}
\centering
\includegraphics[scale=0.5]{fig/sameerdctplot.pdf}
\end{figure}

\end{frame}

%%%%%%%%%%%%%%%%%%%%%%%%%%%%%%%%%%%%%%
%% Frame
%%%%%%%%%%%%%%%%%%%%%%%%%%%%%%%%%%%%%%
\begin{frame}[t]
	\frametitle{2D examples - \st{Lena} Sameer}

\centering
TV reconstruction on left, $\ell_1$ on right.
\begin{figure}
\centering
\includegraphics[scale=0.2]{fig/sameerl1rec.pdf}
\end{figure}

\end{frame}



%%%%%%%%%%%%%%%%%%%%%%%%%%%%%%%%%%%%%%
%% Frame
%%%%%%%%%%%%%%%%%%%%%%%%%%%%%%%%%%%%%%
\begin{frame}[t]
	\frametitle{2D examples - \st{Lena} Sameer}

We get "visaully displeasing high frequency artifacts"
\begin{figure}
\begin{subfigure}{0.5\textwidth}
  \centering
  \includegraphics[height=2in]{fig/sameerTVzoom.pdf}
  \caption{TV solution - $\norm{x^\# -x_0}_{\ell_2} = 0.0159$.}
\end{subfigure}%
\begin{subfigure}{0.5\textwidth}
  \centering
  \includegraphics[height=2in]{fig/sameerl1zoom.pdf}
  \caption{$\ell_1$ solution - $\norm{x^\# -x_0}_{\ell_2} = 0.0903$.}
\end{subfigure}
\end{figure}

\end{frame}


%%%%%%%%%%%%%%%%%%%%%%%%%%%%%%%%%%%%%%
%% Frame
%%%%%%%%%%%%%%%%%%%%%%%%%%%%%%%%%%%%%%
\begin{frame}[t]
	\frametitle{2D examples - \st{Lena} Sameer}

So the $\ell_1$ reconstruction is good, but the TV does work better, at least in natural images.
	
\begin{figure}
\centering
\includegraphics[scale=0.2]{fig/sameerl1rec.pdf}
\end{figure}

\end{frame}




\appendix

%%%%%%%%%%%%%%%%%%%%%%% 1D Examples %%%%%%%%%%%%%%%%%%%%%%%%%%%%%%%%%%

%%%%%%%%%%%%%%%%%%%%%%%%%%%%%%%%%%%%%%
%% Frame
%%%%%%%%%%%%%%%%%%%%%%%%%%%%%%%%%%%%%%

\begin{frame}[t]
\frametitle{1D examples}
\framesubtitle{~~}  %% needed for proper positioning of the logo ...

1D examples shamelessly stolen from
\href{http://compsens.eecs.umich.edu/sensing_tutorial.php}{\textcolor{blue}{http://compsens.eecs.umich.edu/sensing\_tutorial.php}}

In these 1D examples we solve the problem
\begin{align*}
	\min \norm{x}_{\ell_1} \text{ subject to } \norm{Ax - y}_{\ell_2} \leq \epsilon
\end{align*}

using the package \href{http://users.ece.gatech.edu/~justin/l1magic/}{\textcolor{blue}{$\ell_1$-MAGIC}} to perform the optimization.

\end{frame}


%%%%%%%%%%%%%%%%%%%%%%%%%%%%%%%%%%%%%%
%% Frame
%%%%%%%%%%%%%%%%%%%%%%%%%%%%%%%%%%%%%%
\begin{frame}[t]
\frametitle{1D examples - sparse in frequency}
\framesubtitle{~~}  %% needed for proper positioning of the logo ...

\centering
\includegraphics[scale=0.5]{fig/spfreq1D.pdf}

\end{frame}


%%%%%%%%%%%%%%%%%%%%%%%%%%%%%%%%%%%%%%
%% Frame
%%%%%%%%%%%%%%%%%%%%%%%%%%%%%%%%%%%%%%
\begin{frame}[t]
\frametitle{1D examples - sparse in frequency}
\framesubtitle{~~}  %% needed for proper positioning of the logo ...

\centering
\includegraphics[scale=0.5]{fig/spfreq1Drec.pdf}

\end{frame}


%%%%%%%%%%%%%%%%%%%%%%%%%%%%%%%%%%%%%%
%% Frame
%%%%%%%%%%%%%%%%%%%%%%%%%%%%%%%%%%%%%%
\begin{frame}[t]
\frametitle{1D examples - sparse in time}
\framesubtitle{~~}  %% needed for proper positioning of the logo ...

\centering
\includegraphics[scale=0.5]{fig/sptime.pdf}

\end{frame}


%%%%%%%%%%%%%%%%%%%%%%%%%%%%%%%%%%%%%%
%% Frame
%%%%%%%%%%%%%%%%%%%%%%%%%%%%%%%%%%%%%%
\begin{frame}[t]
\frametitle{1D examples - sparse in time}
\framesubtitle{~~}  %% needed for proper positioning of the logo ...

\centering
\includegraphics[scale=0.5]{fig/sptime2.pdf}

\end{frame}


%%%%%%%%%%%%%%%%%%%%%%% 1D Examples End  %%%%%%%%%%%%%%%%%%%%%%%%%%%%%%%%%%


%%%%%%%%%%%%%%%%%%%%%%%%%%%%%%%%%%%%%%
%% Frame
%%%%%%%%%%%%%%%%%%%%%%%%%%%%%%%%%%%%%%
\begin{frame}[t]
\frametitle{1D examples - music}
\framesubtitle{~~}  %% needed for proper positioning of the logo ...


\centering
\includegraphics[scale=0.4]{fig/tstimefreq.pdf}

\end{frame}


%%%%%%%%%%%%%%%%%%%%%%%%%%%%%%%%%%%%%%
%% Frame
%%%%%%%%%%%%%%%%%%%%%%%%%%%%%%%%%%%%%%
\begin{frame}[t]
\frametitle{1D examples - music}
\framesubtitle{~~}  %% needed for proper positioning of the logo ...


\centering
\includegraphics[scale=0.4]{fig/tsfreq.pdf}

\end{frame}


%%%%%%%%%%%%%%%%%%%%%%%%%%%%%%%%%%%%%%
%% Frame
%%%%%%%%%%%%%%%%%%%%%%%%%%%%%%%%%%%%%%
\begin{frame}[t]
\frametitle{1D examples - music}
\framesubtitle{~~}  %% needed for proper positioning of the logo ...


\centering
\includegraphics[scale=0.4]{fig/tstime.pdf}

\end{frame}


%%%%%%%%%%%%%%%%%%%%%%%%%%%%%%%%%%%%%%
%% Frame
%%%%%%%%%%%%%%%%%%%%%%%%%%%%%%%%%%%%%%
\begin{frame}[t]
\frametitle{2D examples - phantom}
\framesubtitle{~~}  %% needed for proper positioning of the logo ...

Phantom image, 4.5 SNR and only sampling 1\% of the data

\centering
\includegraphics[scale=0.5]{fig/phantom100.pdf}

\end{frame}

\iftrue

%%%%%%%%%%%%%%%%%%%%%%%%%%%%%%%%%%%%%%
%% Frame
%%%%%%%%%%%%%%%%%%%%%%%%%%%%%%%%%%%%%%
\begin{frame}[t]
	\frametitle{2D examples - Some more}
\pause
	
\begin{figure}
\centering
\includegraphics[scale=0.2]{fig/sameerl1rec.pdf}
\end{figure}

\end{frame}

%%%%%%%%%%%%%%%%%%%%%%%%%%%%%%%%%%%%%%
%% Frame
%%%%%%%%%%%%%%%%%%%%%%%%%%%%%%%%%%%%%%
\begin{frame}[t]
	\frametitle{2D examples - Sameer's long lost twin brother?}

$\norm{x^\# -x_0}_{\ell_2} = 0.1092$, $\ell_1$ reconstruction, no noise.

\begin{figure}
\centering
\includegraphics[scale=0.2]{fig/oscar.pdf}
\end{figure}


\end{frame}

%%%%%%%%%%%%%%%%%%%%%%%%%%%%%%%%%%%%%%
%% Frame
%%%%%%%%%%%%%%%%%%%%%%%%%%%%%%%%%%%%%%
\begin{frame}[t]
	\frametitle{2D examples - Sameer's long lost cousin?}

$\norm{x^\# -x_0}_{\ell_2} = 0.1553$, SNR=4.5, 1:10 sampling to image size ratio.

\begin{figure}
\centering
\includegraphics[scale=0.5]{fig/sloth.pdf}
\end{figure}


\end{frame}

\fi

%%%%%%%%%%%%%%%%%%%%%%%%%%%%%%%%%%%%%%
%% Frame
%%%%%%%%%%%%%%%%%%%%%%%%%%%%%%%%%%%%%%
\begin{frame}[t]
	\frametitle{2D examples - Rachel}

\begin{figure}
\centering
\includegraphics[scale=0.6]{fig/rachel.pdf}
\end{figure}

All else equal, smaller images don't perform as well. We need a high ratio of image size to number of samples
to be able to reconstruct it well.


\end{frame}

%%%%%%%%%%%%%%%%%%%%%%%%%%%%%%%%%%%%%%
%% Frame
%%%%%%%%%%%%%%%%%%%%%%%%%%%%%%%%%%%%%%
\begin{frame}[t]
	\frametitle{Compressed Sensing is Awesome!}

\begin{figure}
\centering
\includegraphics[scale=0.5]{fig/trex.pdf}
\end{figure}


\end{frame}

\end{document}


