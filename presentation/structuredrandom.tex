%%%%%%%%%%%%%%%%%%%%%%%%%%%%%%%%%%%%%%
%% Frame
%%%%%%%%%%%%%%%%%%%%%%%%%%%%%%%%%%%%%%

\begin{frame}[t]
	\frametitle{Structured Random Matrices and the DFR Framework}
	\framesubtitle{~~}  %% needed for proper positioning of the logo ...

	Do et al. (2012) introduced the structured random matrix framework for compressed sensing,
	which is a way of generating sensing matrices $\Phi$ that are incoherent with any basis $\Psi$. Let 

	\centering
	$	\Phi = \sqrt{\frac{N}{M}}DFR$

	\begin{itemize}
	\item $D\in \mathbb{R}^{M\times N}$ (output randomizer) is a subsampling matrix/operator which does one of the following:
		\begin{itemize}
		\item Randomly selects a subset of the entries of a vector (and possibly permutes it) We call this subsampling.
		\item Randomly pre-modulated summation operator as introduced in Romberg, 2008.
		\end{itemize}
	\item $F\in\mathbb{R}^{N\times N}$ is an orthonormal matrix. We usually use a fast transform like the FFT or 
		fast Walsh-Hadamard transform for this.
	\item $R\in\mathbb{R}^{N\times N}$ (input randomizer) is one of the following:
		\begin{itemize}
			\item A uniform random permutation matrix
			\item A diagonal matrix whose entries are Bernoulli random variables with probability $1/2$
		\end{itemize}
	\end{itemize}

\end{frame}

%%%%%%%%%%%%%%%%%%%%%%%%%%%%%%%%%%%%%%
%% Frame
%%%%%%%%%%%%%%%%%%%%%%%%%%%%%%%%%%%%%%
\begin{frame}[t]
	\frametitle{Randomly pre-modulated summation (RPMS)}
	\framesubtitle{~~}  %% needed for proper positioning of the logo ...
	
	Within the DFR framework one, of the restruction or subsampling operators is the
	randomly pre-modulated summation operator. It works as follows:

	Given two vectors $x$ and $\epsilon$, we compute
	\begin{equation}
		y_k = \sqrt{\frac{M}{N}}\sum_{t\in B_k}\epsilon_t x_t
	\end{equation}
	where
	$B_k = \{(k-1)N/M + 1, \ldots, kN/M  \}, k = 1,\ldots,m$ if $M$ evenly divides $N$.

\end{frame}

